\documentclass[12,a4paper]{article}
\usepackage{classiccomputing}

\newcommand\postertype{lightgray}

\begin{document}

\makecclogo

\author{} % owner of the device, also used as "Author" in PDF export
\title{RIPterm / RIPscrip}
\def\subtitle{Vektorgrafik für Mailboxen}
\def\introduction{Markteinführung 1992}
\makeheader{35}{40} % title size 35, subtitle size 40

%\includeimage {images/Livingston_Portmaster3_top.png}{0.4}

\makebullets{
    DOS-Software für Einwahl in Mailboxen \newline
    Auflösung: EGA 640 x 350  \newline
    Statt ASCII/ANSI-Text werden Vektorgrafikbefehle übertragen \newline
}

\makemain{
    Terminalprogramme für Einwahl in Mailboxen über das Telefonnetz waren üblicherweise Textbasiert und
    existierten für alle unterschiedlichen Betriebssysteme und Rechnerarchitekturen.

    Die Amerikanische Firma {\em TeleGrafX Communications} entwickelte eine Sprache zur Beschreibung und
    Übertragung von Vektorgrafiken (RIPscrip) sowie das Terminalprogramm RIPterm.  Dies wurde von spezieller
    Mailboxsoftware dazu benutzt, dem Benutzer grafische Mailbox-Menüs und Medienkunst zu präsentieren.
}

\includegraphics[width=0.9\linewidth]{images/HOUND.RIP.png}

\makefooter

\end{document}
