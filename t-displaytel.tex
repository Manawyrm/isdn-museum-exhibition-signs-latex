\documentclass[12,a4paper]{article}
\usepackage{classiccomputing}

\newcommand\postertype{lightgray}

\begin{document}

\makecclogo

\author{LaF0rge} % owner of the device, also used as "Author" in PDF export
\title{T-DisplayTel}
\def\subtitle{ISDN-Telefon mit BTX}
\def\introduction{Markteinführung 1995}
\makeheader{40}{40} % title size 60, subtitle size 40

\includeimage {images/t-displaytel.jpg}{0.4}

\makebullets{
    Land: Deutschland \newline
    Preis: unbekannt \newline
   
    Architektur: 16 Bit (V35, kompatibel zu 8088) \newline
    CPU: NEC uPD70236 \newline
    % RAM: 
    % ROM: 
    % Grafik: 
    % Sound:

    Hersteller: Unbekannt \newline
    BTX-Dekoder IC: LOEWE \newline
    ISDN-Chipsatz: Siemens ISAC + HSCX \newline
}

\makemain{
    Das T-DisplayTel ist ein ISDN-Tischtelefon mit eingebautem BTX-Dekoder.\newline
    
    Die Übertragung von BTX über ISDN war relativ selten, da fast alle BTX-Dekoder mit analogen Modems
    ausgestattet waren.  Es wird hierbei T.70 NL über X.75 im B-Kanal verwendet.\newline
    
    BTX-Dekoder hatten üblicherweise Röhrenbildschirme bzw. wurden am Fernseher angeschlossen.
    LCD-Flachbildschirme wie in diesem Gerät sind eine Seltenheit.

    Es wird vermutet, dass die T-DisplayTel relativ selten verkauft wurden, da sie mutmasslich teuer waren und erst auf den Markt kamen, als BTX bereits seine Bedeutung weitgehend verloren hatte.
}

\makefooter

\end{document}
