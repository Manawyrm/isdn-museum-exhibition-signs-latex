\documentclass[12,a4paper]{article}
\usepackage{classiccomputing}

\newcommand\postertype{lightgray}

\begin{document}

\makecclogo

\author{Manawyrm} % owner of the device, also used as "Author" in PDF export
\title{T-View 100}
\def\subtitle{ISDN-Bildtelefon}
\def\introduction{Markteinführung 1997}
\makeheader{40}{40} % title size 60, subtitle size 40

\includeimage {images/tview1200.png}{0.4}

\makebullets{
    Land: Deutschland \newline
    Preis: 1000 DM \newline
   
    Architektur: 32 Bit \newline
    % CPU:
    % RAM: 
    % ROM: 
    % Grafik: 
    % Sound:
    2x Motorola 68000 CPU \newline

    Hersteller: Siemens / Telekom \newline
    Codecs: H.263, H.261, G.722 \newline
    Auflösung: 352 x 288 oder 176 x 144 \newline
}

\makemain{
    Das T-View 100 nutzt Kanalbündelung um Bild/Ton mit 128kBit/s zu übertragen.\newline
    
    Es nutzt dabei den Standard H.320 zur Aushandlung von Codecs (ähnlich H.323/NetMeeting).\newline
    
    Als Basistelefon wurde das Mainboard vom Siemens ProfiTel 70, allerdings mit anderer
    Firmware verwendet. Videoein- \& ausgabe erfolgt mittels Composite-Video.\newline
    Ein Philips SAA7111 (wie auf TV-Karten) kümmert sich um die Digitalisierung.\newline
    
    RS232-Schnittstelle ermöglicht Anschluss von Zusatzgeräten, z.B. Stroboskoplampen
    für gehörlose Menschen.\newline
    
    Manche VHS-Rekorder liessen sich über das T-View fernbedienen (z.B. um ein
    Überwachungsvideo aus der Ferne abzurufen).\newline
    
    Es gab öffentliche Webcams per H.320.\newline
}

\makefooter

\end{document}