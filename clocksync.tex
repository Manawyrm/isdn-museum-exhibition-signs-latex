\documentclass[12,a4paper]{article}
\usepackage{classiccomputing}

\newcommand\postertype{lightgray}

\begin{document}

\makecclogo

\author{} % owner of the device, also used as "Author" in PDF export
\title{Fax/Modem über VoIP}
\def\subtitle{Asynchrone und synchrone Netze}
\def\introduction{Auf's richtige Timing kommt es an}
\makeheader{50}{30} % title size 35, subtitle size 40

%\includeimage {images/icE1usb-usb_side.jpg}{0.4}

\makebullets{
    VoIP über SIP nutzt genau wie ISDN oft den Sprachcodec G.711a/alaw zur Audioübertragung.
    Trotzdem funktioniert die Nutzung von Fax/Modem über ISDN wesentlich besser als bei den meisten VoIP-Systemen.
    \newline\newline
    Der Grund dafür liegt im Timing: Bei ISDN wird ein Takt vom Netz vorgegeben. Dieser Takt wird zur Übertragung der Bits genutzt, Telefonanlagen, Geräte und Computer synchronisieren sich auf genau diesen Takt.
    Wenn ein analoges Endgerät angeschlossen wird, wird es mit diesem Takt (oder einer abgeleiteten Frequenz) abgetastet.
    \newline\newline
    Bei VoIP stellt jedes Endgerät einen eigenen Takt her (durch lokale Oszillatoren im Gerät). Bei VoIP kommt es daher zu Frequenzdrift. \newline
    Beispiel: Statt den geforderten 8000 Hz Abtastrate wird mit 8001 oder 7999 Hz gearbeitet. Damit entsteht ein weiterer Messwert pro Sekunde (oder es fehlt ein Messwert pro Sekunde).
    Diese Unterschiede werden entweder durch Resampling ausgeglichen oder führen zu Pufferüber- oder unterläufen.
    Beides ist bei Sprachkommunikation nicht tragisch, führt bei Modemverbindungen aber zu Verbindungsabbrüchen.
    \newline\newline
    Weitere Effekte wie Jitter (Phasenrauschen) werden beim Design von VoIP-Systemen häufig ebenfalls ignoriert, weil sie für Sprache fast irrelevant sind.
}

\makemain{
    
}

\makefooter

\end{document}