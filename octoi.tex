\documentclass[12,a4paper]{article}
\usepackage{classiccomputing}

\newcommand\postertype{lightgray}

\begin{document}

\makecclogo

\author{} % owner of the device, also used as "Author" in PDF export
\title{OCTOI-Protokoll}
\def\subtitle{ISDN über das Internet}
\def\introduction{Die Neuerung}
\makeheader{40}{40} % title size 35, subtitle size 40

\includeimage {images/icE1usb-usb_side.jpg}{0.4}

\makebullets{
    Um ISDN (und andere TDM-Protokolle) \newline
    wieder nutzbar zu machen, wurde das  \newline
    Osmocom Community TDM over IP-Protokoll  \newline
    entwickelt. \newline\newline
    Mit OCTOI wird ein Primärmultiplexanschluss (2 MBit/s) \newline
    in UDP-Pakete eingepackt, welche über das Internet  \newline übertragen werden können.

    Damit können interessierte Nutzer per Internet an eine virtuelle Vermittlungsstelle in einem Rechenzentrum angeschlossen werden. \newline
    Diese virtuelle Vermittlungsstelle (bei uns auch DIVF genannt) vermittelt dann zwischen allen Teilnehmern sowie einigen Diensten. \newline

    Das OCTOI-Protokoll vermeidet die Übertragung von ungenutzten Zeitschlitzen und vermindert so die Bandbreitennutzung im Ruhezustand. \newline

    Implementiert wurde OCTOI bisher in osmo-e1d, einem Linux-Programm. osmo-e1d kann entweder mit einem icE1usb (oben gezeigt) sprechen und eine echte E1-Leitung (Primärmultiplex) per USB bereitstellen oder ein virtuelles DAHDI-E1-Gerät im Linux-Kernel anmelden (trunkdev-Modus). \newline

    Zur Nutzung wird ein Linux-Computer benötigt. Kleine Single-Board-Computer (Raspberry Pi, NanoPi, etc.) sind mehr als ausreichend.
}

\makemain{
    
}

\makefooter

\end{document}